The findings of this study demonstrate that the number of parallel TCP connections has a significant impact on page load time, 
although there is a diminishing return as the number of connections increases excessively. Based on the results, the default 
choice of six concurrent TCP connections made by browsers appears to be a reasonable selection.

Additionally, caching mechanisms, including the 'Expiration,' 'Validation,' and 'Heuristic' strategies, consistently contribute 
to improved performance. The results clearly indicate the overall benefits of caching, highlighting its importance in reducing 
page load time and enhancing the user experience.

The analysis of different HTTP versions revealed that each new version outperforms its predecessor, regardless of the caching 
mechanism employed. These performance improvements can be attributed to the enhancements introduced in each new version, 
emphasizing the significance of utilizing the latest HTTP protocols.

Regarding the performance analysis of the analyzed websites, it can be concluded that the fastest website is 'MIT,' followed 
by 'Harvard' and 'Apple.'

Although the direct impact of warm-up time on the analyzed websites was found to be not highly relevant, primarily due to 
limited available data, notable observations were made regarding its influence on the RPS (Requests Per Second) and TPR SD (Standard Deviation) 
metrics. Specifically, it was observed that a warm-up time of 6 seconds, comprising approximately 30\% of the total measurement
time, resulted in a peak in RPS and a decrease in TPR SD. These findings suggest that allocating an adequate warm-up period for servers can yield positive implications for their overall performance.

In conclusion, this study provides valuable insights into various aspects of web technologies. The optimal number of parallel TCP 
connections, the importance of caching mechanisms, the superiority of newer HTTP versions, and the potential impact of warm-up 
time allocation for the websites performance assessment.