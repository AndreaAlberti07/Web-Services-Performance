This report presents an analysis of web services, focusing on their implementation aspects and the impact on their performance. 
The study examines several factors that influence performance, including the number of parallel TCP connections, caching mechanisms, 
HTTP versions, and warm-up time allocation.
The primary metric used to investigate the impact of these factors on the performance is the page load time (PLT). Additionally, the study aims to compare the performance of different websites under various conditions, 
utilizing metrics such as Requests Per Second (RPS), Time Per Request (TPR) and its Standard Deviation (SD).
Several tools were employed for data collection and analysis, including the Apache Benchmark, the Web Development Tools integrated into Mozilla Firefox, 
and the h2load tool. The experiments were performed multiple times within the same day to ensure accurate and consistent results.
The study highlights the optimal number of parallel TCP connections, showing the importance of caching mechanisms and  
the superiority of newer HTTP versions. Furthermore, the study identifies the potential impact of warm-up time allocation on web service performance.
Overall, this research contributes to the understanding and improvement of web service implementation and performance. 
